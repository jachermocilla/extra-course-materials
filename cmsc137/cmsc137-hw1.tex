\documentclass[a4paper, 11pt,oneside]{article}
\usepackage[
  top=1.5cm,
  bottom=1cm,
  left=2cm,
  right=1.5cm,
  headheight=25.22153pt, % as per the warning by fancyhdr
  includehead,includefoot,
  heightrounded, % to avoid spurious underfull messages
]{geometry} 
\usepackage{hyperref}
\usepackage[T1]{fontenc}
\usepackage{microtype}
\usepackage{fancyhdr}
\usepackage{fancyvrb}
\usepackage{lipsum}
\usepackage{url}
\usepackage{listings}
\usepackage{lastpage}
\usepackage{enumitem}
\usepackage{datetime}

\settimeformat{hhmmsstime}
\yyyymmdddate

\pagestyle{fancy}
\fancyhf{} % clear all fields

\pagestyle{fancy}
\lhead{CMSC 137: Data Communications and Networking \\ First Semester 2018-2019}
\rhead{Institute of Computer Science \\ University of the Philippines Los Banos}
\rfoot{JACHermocilla (jchermocilla@up.edu.ph)}
%\cfoot{Enjoy!:)}
\cfoot{\thepage\ of \pageref{LastPage}}
\lfoot{Revision: \today\ \currenttime}
%\rfoot{https://jachermocilla.org/teaching/125}
\renewcommand{\headrulewidth}{0.4pt}
\renewcommand{\footrulewidth}{0.4pt}

\begin{document}

\begin{center}
   {\LARGE \textbf{Homework 1: Exploring TCP/IP Protocols }}
\end{center}

\section*{Objectives}
   At the end of this activity, you should be able to 
   understand the operation of different protocols by anaylizing captured packets using Wireshark.

\section{Introduction}
The Internet is powered by the TCP/IP protocol suite, a layered architecture composed of five layers: Physical(L1), Data Link(L2), Network/Internet(L3), Transport(L4), and Application(L7). Each layer use services provided by the layer below it and it provides services to the layer above it. The services provided by each layer, when formalized for possible implementation and approved by a recognized organization or body, becomes a \textit{protocol}. A protocol can then be implemented in an actual working and usable system or network.

L1 and L2 are usually combined into one as the Network Access Layer to provide the underlying LAN or WAN standard. Example standards are IEEE 802.3(Wired Ethernet) and IEEE 802.11(Wireless). Its main function is reliable delivery of data (through flow control and error control) between the nodes at the two ends of the physical link. Nodes can be hosts(e.g. computers), routers, or switches.

L3 is responsible moving \textit{packets} between two hosts that are located in different LANs or WANs. Its main services include addressing, routing and forwarding, and fragmentation. The \textit{Internet Protocol (IP)} is the main protocol for this layer. Other protocols include the \textit{Internet Control Message Protocol (ICMP)} and the \textit{Address Resolution Protocol (ARP)}

L4 is responsible for the delivery of data from one process(a client) in one host to another process(a server) in another host. A \textit{process}, in the context of operating system, is a running program. For example, a web browser running on your laptop uses the services of L4 to download a web page from a web server running in a computer located in a data center somewhere. L4 thus is responsible for process-to-process delivery of data with functions such as addressing, multiplexing/demultiplexing, congestion control, flow control, and error control. Protocols in L4 include \textit{Transmission Control Protocol (TCP)}, \textit{User Datagram Protocol (UDP)}, and \textit{Stream Control Transmission Protocol (SCTP)}.

Lastly, L7 is responsible for user-specific services such as remote access, file transfer, the World Wide Web, electronic mail, name resolution, network configuration, etc. Some L7 protocols include \textit{Telnet}, \textit{Secure Shell (SSH)}, \textit{File Transfer Protocol (FTP)}, \textit{HyperText Transfer Protocol(HTTP)}, \textit{Simple Mail Transfer Protocol(SMTP)}, \textit{Domain Name System (DNS)}, and \textit{Dynamic Host Configuration Protocol (DHCP)}.   

In this homework, you will explore the different protocols for each layer.  

\section{Prerequisites}
You should review the following chapters from the textbook:
\begin{enumerate}
	\item{Chapter 18 (Introduction to Network Layer)} 
	\item{Chapter 19 (Network-Layer Protocols)} 
	\item{Chapter 23 (Introduction to Transport Layer)} 
	\item{Chapter 24 (Transport-Layer Protocols)} 
\end{enumerate}
You should also be familiar with the basic use of Wireshark. All packet captures for this homework are available at:
\begin{verbatim}
https://github.com/jachermocilla/extra-course-materials/tree/master/cmsc137/wireshark
\end{verbatim}


\section{Deliverables and Credit}


\section{Tasks}

\subsection*{Task 1: Address Resolution Protocol (3 points)} 

\href{https://github.com/jachermocilla/extra-course-materials/raw/master/cmsc137/wireshark/arp.pcapng}{ARP Download}

\noindent\fbox{
	\parbox{\textwidth}{
	\textbf{QUESTIONS}(Provide screen shots to support your answer):
	\begin{enumerate}[itemsep=0pt,parsep=0pt]
	\item{What data structure is used for the implementation of environment variables?}
	\item{Are environment variables unique for each process or shared by all processes?}
	\item{What are the functions used to set and get an environment variable?}
	\item{Examine \texttt{kernel/console/console.c}. What console commands use the functions in question 3?} 
	\end{enumerate}	
	}
}

\subsection*{Task 2: Internet Protocol (3 points)} 

\href{}{IP Download}

\noindent\fbox{
	\parbox{\textwidth}{
	\textbf{QUESTIONS}(Provide screen shots to support your answer):
	\begin{enumerate}[itemsep=0pt,parsep=0pt]
	\item{What data structure is used for the implementation of environment variables?}
	\item{Are environment variables unique for each process or shared by all processes?}
	\item{What are the functions used to set and get an environment variable?}
	\item{Examine \texttt{kernel/console/console.c}. What console commands use the functions in question 3?} 
	\end{enumerate}	
	}
}

\subsection*{Task 3: Internet Control Message Protocol (3 points)} 

\href{https://github.com/jachermocilla/extra-course-materials/raw/master/cmsc137/wireshark/ping.pcapng}{ICMP Download}

\noindent\fbox{
	\parbox{\textwidth}{
	\textbf{QUESTIONS}(Provide screen shots to support your answer):
	\begin{enumerate}[itemsep=0pt,parsep=0pt]
	\item{What data structure is used for the implementation of environment variables?}
	\item{Are environment variables unique for each process or shared by all processes?}
	\item{What are the functions used to set and get an environment variable?}
	\item{Examine \texttt{kernel/console/console.c}. What console commands use the functions in question 3?} 
	\end{enumerate}	
	}
}

\subsection*{Task 4: User Datagram Protocol (3 points)} 

\href{}{UDP Download}

\noindent\fbox{
	\parbox{\textwidth}{
	\textbf{QUESTIONS}(Provide screen shots to support your answer):
	\begin{enumerate}[itemsep=0pt,parsep=0pt]
	\item{What data structure is used for the implementation of environment variables?}
	\item{Are environment variables unique for each process or shared by all processes?}
	\item{What are the functions used to set and get an environment variable?}
	\item{Examine \texttt{kernel/console/console.c}. What console commands use the functions in question 3?} 
	\end{enumerate}	
	}
}

\subsection*{Task 5: Traceroute (3 points)} 

\href{https://github.com/jachermocilla/extra-course-materials/raw/master/cmsc137/wireshark/traceroute.pcapng}{Traceroute Download}

\noindent\fbox{
	\parbox{\textwidth}{
	\textbf{QUESTIONS}(Provide screen shots to support your answer):
	\begin{enumerate}[itemsep=0pt,parsep=0pt]
	\item{What data structure is used for the implementation of environment variables?}
	\item{Are environment variables unique for each process or shared by all processes?}
	\item{What are the functions used to set and get an environment variable?}
	\item{Examine \texttt{kernel/console/console.c}. What console commands use the functions in question 3?} 
	\end{enumerate}	
	}
}

\subsection*{Task 6: Transmission Control Protocol (3 points)} 

\href{}{TCP Download}

\noindent\fbox{
	\parbox{\textwidth}{
	\textbf{QUESTIONS}(Provide screen shots to support your answer):
	\begin{enumerate}[itemsep=0pt,parsep=0pt]
	\item{What data structure is used for the implementation of environment variables?}
	\item{Are environment variables unique for each process or shared by all processes?}
	\item{What are the functions used to set and get an environment variable?}
	\item{Examine \texttt{kernel/console/console.c}. What console commands use the functions in question 3?} 
	\end{enumerate}	
	}
}

\subsection*{Task 7: Telnet (3 points)} 

\href{https://github.com/jachermocilla/extra-course-materials/raw/master/cmsc137/wireshark/telnet.pcapng}{Telnet Download}

\noindent\fbox{
	\parbox{\textwidth}{
	\textbf{QUESTIONS}(Provide screen shots to support your answer):
	\begin{enumerate}[itemsep=0pt,parsep=0pt]
	\item{What data structure is used for the implementation of environment variables?}
	\item{Are environment variables unique for each process or shared by all processes?}
	\item{What are the functions used to set and get an environment variable?}
	\item{Examine \texttt{kernel/console/console.c}. What console commands use the functions in question 3?} 
	\end{enumerate}	
	}
}

\subsection*{Task 8: Secure Shell (3 points)} 

\href{https://github.com/jachermocilla/extra-course-materials/raw/master/cmsc137/wireshark/ssh.pcapng}{SSH Download}

\noindent\fbox{
	\parbox{\textwidth}{
	\textbf{QUESTIONS}(Provide screen shots to support your answer):
	\begin{enumerate}[itemsep=0pt,parsep=0pt]
	\item{What data structure is used for the implementation of environment variables?}
	\item{Are environment variables unique for each process or shared by all processes?}
	\item{What are the functions used to set and get an environment variable?}
	\item{Examine \texttt{kernel/console/console.c}. What console commands use the functions in question 3?} 
	\end{enumerate}	
	}
}

\subsection*{Task 9: HyperText Transfer Protocol (3 points)} 

\href{https://github.com/jachermocilla/extra-course-materials/raw/master/cmsc137/wireshark/http.pcapng}{HTTP Download}

\noindent\fbox{
	\parbox{\textwidth}{
	\textbf{QUESTIONS}(Provide screen shots to support your answer):
	\begin{enumerate}[itemsep=0pt,parsep=0pt]
	\item{What data structure is used for the implementation of environment variables?}
	\item{Are environment variables unique for each process or shared by all processes?}
	\item{What are the functions used to set and get an environment variable?}
	\item{Examine \texttt{kernel/console/console.c}. What console commands use the functions in question 3?} 
	\end{enumerate}	
	}
}

\subsection*{Task 10: HyperText Transfer Protocol (Secure) (3 points)} 

\href{https://github.com/jachermocilla/extra-course-materials/raw/master/cmsc137/wireshark/https.pcapng}{HTTPS Download}

\noindent\fbox{
	\parbox{\textwidth}{
	\textbf{QUESTIONS}(Provide screen shots to support your answer):
	\begin{enumerate}[itemsep=0pt,parsep=0pt]
	\item{What data structure is used for the implementation of environment variables?}
	\item{Are environment variables unique for each process or shared by all processes?}
	\item{What are the functions used to set and get an environment variable?}
	\item{Examine \texttt{kernel/console/console.c}. What console commands use the functions in question 3?} 
	\end{enumerate}	
	}
}

\subsection*{Task 11: Domain Name System (3 points)} 

\href{https://github.com/jachermocilla/extra-course-materials/raw/master/cmsc137/wireshark/dns.pcapng}{DNS Download}

\noindent\fbox{
	\parbox{\textwidth}{
	\textbf{QUESTIONS}(Provide screen shots to support your answer):
	\begin{enumerate}[itemsep=0pt,parsep=0pt]
	\item{What data structure is used for the implementation of environment variables?}
	\item{Are environment variables unique for each process or shared by all processes?}
	\item{What are the functions used to set and get an environment variable?}
	\item{Examine \texttt{kernel/console/console.c}. What console commands use the functions in question 3?} 
	\end{enumerate}	
	}
}

\subsection*{Task 12: File Transfer Protocol (3 points)} 

\href{https://github.com/jachermocilla/extra-course-materials/raw/master/cmsc137/wireshark/ftp.pcapng}{FTP Download}

\noindent\fbox{
	\parbox{\textwidth}{
	\textbf{QUESTIONS}(Provide screen shots to support your answer):
	\begin{enumerate}[itemsep=0pt,parsep=0pt]
	\item{What data structure is used for the implementation of environment variables?}
	\item{Are environment variables unique for each process or shared by all processes?}
	\item{What are the functions used to set and get an environment variable?}
	\item{Examine \texttt{kernel/console/console.c}. What console commands use the functions in question 3?} 
	\end{enumerate}	
	}
}

\subsection*{Task 13: Dynamic Host Configuration Protocol (3 points)} 

\href{https://github.com/jachermocilla/extra-course-materials/raw/master/cmsc137/wireshark/dhcp.pcapng}{DHCP Download}

\noindent\fbox{
	\parbox{\textwidth}{
	\textbf{QUESTIONS}(Provide screen shots to support your answer):
	\begin{enumerate}[itemsep=0pt,parsep=0pt]
	\item{What data structure is used for the implementation of environment variables?}
	\item{Are environment variables unique for each process or shared by all processes?}
	\item{What are the functions used to set and get an environment variable?}
	\item{Examine \texttt{kernel/console/console.c}. What console commands use the functions in question 3?} 
	\end{enumerate}	
	}
}

\subsection*{Task 14: Simple Mail Transfer Protocol (3 points)} 

\href{https://github.com/jachermocilla/extra-course-materials/raw/master/cmsc137/wireshark/smtp.pcapng}{SMTP Download}

\noindent\fbox{
	\parbox{\textwidth}{
	\textbf{QUESTIONS}(Provide screen shots to support your answer):
	\begin{enumerate}[itemsep=0pt,parsep=0pt]
	\item{What data structure is used for the implementation of environment variables?}
	\item{Are environment variables unique for each process or shared by all processes?}
	\item{What are the functions used to set and get an environment variable?}
	\item{Examine \texttt{kernel/console/console.c}. What console commands use the functions in question 3?} 
	\end{enumerate}	
	}
}

\end{document}
